\documentclass{article}
\usepackage[utf8]{inputenc}
\usepackage[brazil]{babel}
\usepackage{mathtools}
\usepackage[colorlinks, linkcolor=blue, urlcolor=blue, citecolor=blue]{hyperref}
\usepackage[a4paper, left=20mm, right=20mm, top=20mm, bottom=20mm]{geometry}

\title{\textbf{Números primos \\
        \large INE5429 - Segurança em Computação}}
\author{
    Caique Rodrigues Marques \\
    {\texttt{c.r.marques@grad.ufsc.br}}
}
\date{}

\begin{document}

\maketitle

\begin{itemize}
    \item O teste de primalidade de Fermat define que se $p$ é primo e $a$ não
    é divisível por $p$, então temos: 
    \begin{equation}\label{eq:fermat}
        a^{p-1} \equiv 1 \pmod{p}.
    \end{equation}
    Para testar se $p$ é primo, então vários $a$'s aleatórios e não divisíveis
    por $p$ são coletados. Se a igualdade em \ref{eq:fermat} não for válida com
    os valores definidos de $a$ e $p$, então é dito que $p$ é composto. Por
    outro lado, se a igualdade for válida para um ou mais valores de $a$, então
    $p$ é um possível primo. No entanto, o teste aprova determinados
    pseudoprimos\footnotemark[1], principalmente os números de
    Carmichael\footnotemark[2].

    \footnotetext[1]{Os números $p$ que satisfazem a condição em
    \ref{eq:fermat}, mas não são primos, são chamados de pseudoprimos de
    Fermat.}
    \footnotetext[2]{Número de Carmichael é um número composto $p$ que satisfaz
    a relação em \ref{eq:fermat} para todos os inteiros $a$ que são coprimos a
    $p$.}
    
    \item O teste de primalidade de Miller-Rabin é um aperfeiçoamento ao teste
    de primalidade de Fermat. Dado um $n$ primo e $n > 2$, então $n-1$ vai ser
    par e pode ser escrito como $2^{s}d$, onde $s$ e $d$ são positivos inteiros
    e $d$ é ímpar. Podemos verificar que $n$ não é primo, se para todo $0 \leq
    r \leq s-1$ temos
    \begin{equation*}
        a^{d} \not\equiv 1 \pmod{n} \land a^{2^{r}d} \not\equiv -1 \pmod{n},
    \end{equation*}
    então $n$ não é primo, caso contrário, $n$ possivelmente é primo. Vale
    notar que $a$ é uma testemunha de que $n$ é composto, assim, $a$'s são
    escolhidos aleatoriamente dentre o intervalo $1 < a < n-1$.
    
    \item Para o caso do teste de Miller-Rabin, apesar de possuir uma
    complexidade maior e, consequentemente, possuir um maior tempo de execução,
    ele possui melhor precisão em detectar possíveis primos porque independe do
    valor de entrada, conseguindo uma probabilidade de $\frac{1}{2}$ de
    detectar um número composto. Ao contrário do caso do teste de Fermat, ele
    consegue atingir a mesma probabilidade se o valor de entrada não for um
    número de Carmichael, portanto, ele acaba tendo um caso claro de
    falso-positivo.

    \item O arquivo \texttt{primality\_test.py} implementa ambos os testes de
    primalidade. Ambos podem ser usados da seguinte maneira:
    \begin{verbatim}
        $ python3
        >>> from primality_test import fermat, miller_rabin
        >>> miller_rabin(64380800680355443923012985496149269915138610, 100)
        True
        >>> fermat(12764787846358441471, 100)
        True
    \end{verbatim}

    O algoritmo de Miller-Rabin usa exponenciação modular, possuindo uma
    complexidade de $O(k\log^{3}n)$, onde $k$ é o número de testes realizados.
    O algoritmo de Fermat, que também usa exponenciação modular, possui uma
    complexidade de
    $O(k\times\log^{2}n\times\log(\log(n))\times\log(\log(\log(n))))$, onde $k$
    é o número de testes realizados.
    
    \item O arquivo \texttt{find\_primes.py} é um simples programa que executa
    testes de primalidade, procurando por números primos com bits de tamanho
    100, 200, 300, ..., 10000. A saída gerada apresenta o tempo para achar o
    número primo, o número de bits e o inteiro primo encontrado. Como é
    possível notar, o programa demora mais para achar primos com bits maiores
    que 2000, devido às várias exponenciações e operações modulares que são
    executadas.
\end{itemize}

\end{document}
