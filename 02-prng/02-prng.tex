\documentclass{article}
\usepackage[utf8]{inputenc}
\usepackage[brazil]{babel}
\usepackage{mathtools}
\usepackage[colorlinks, linkcolor=blue, urlcolor=blue, citecolor=blue]{hyperref}
\usepackage[a4paper, left=20mm, right=20mm, top=20mm, bottom=20mm]{geometry}

\title{\textbf{Geradores de números pseudoaleatórios \\
        \large INE5429 - Segurança em Computação}}
\author{
    Caique Rodrigues Marques \\
    {\texttt{c.r.marques@grad.ufsc.br}}
}
\date{}

\begin{document}

\maketitle
\section*{Os Algoritmos}
    \begin{itemize}
        \item O gerador congruente linear (\textit{LCG} na sigla em inglês) é
            um algoritmo que gera uma sequência de números pseudoaleatórios,
            calculados com uma função linear em trecho. O gerador é definido
            pela seguinte relação de recorrência:
        \begin{align*}
            X_{n+1} = (aX_{n} + c) \mod{m}
        \end{align*}
        onde, 
        \begin{itemize}
            \item $X$ é a sequência de números pseudoaleatórios;
            \item $m$ é o módulo;
            \item $a$ é o multiplicador;
            \item $c$ é o incrementador;
            \item $X_{0}$ é o valor inicial.
        \end{itemize}
        As constantes $m$, $a$, $c$ e $X_{0}$ são valores inteiros que
            especificam o gerador.
        
        A complexidade do LCG, no pior caso é definido por $O(n)$, porque o
            algoritmo apenas realizará cálculos, o que leva $O(1)$, fazendo
            isso $n$ vezes.
        
        \item O gerador de número aleatório de Park-Miller é um tipo de gerador
            congruente linear que opera com grupos multiplicativos de inteiros
            módulo $n$. A fórmula geral, similar à do \textit{LCG} (com $c =
            0$), é a seguinte:
        \begin{align*}
            X_{k+1} = gX_{k} \mod{n},
        \end{align*}
        onde o módulo $n$ é um número primo ou uma potência de um número primo,
            o multiplicador $g$ é um elemento de alta ordem multiplicativa $n$
            e a semente $X_{0}$ é coprimo a $n$. Em outras palavras, o gerador
            de Park-Miller é um LCG onde $c = 0$.
        
        A complexidade do gerador de Park-Miler, no pior caso é definido por
            $O(n)$, assim como no LCG, o algoritmo apenas realizará cálculos, o
            que leva $O(1)$, fazendo isso $n$ vezes.
        
        \item O Xorshift é um gerador de números pseudoaleatório, sendo um
            subconjunto da função \textit{linear-feedback shift registers}
            (LFSRs). O algoritmo gera os números de uma sequência fazendo
            operação de ou-exclusivo (XOR) e deslocamentos de bit à esquerda ou
            à direita.
        
        A complexidade do Xorshift, no pior caso é definido por $O(n)$, pois o
            algoritmo apenas realizará diversos cálculos, o que leva $O(1)$,
            fazendo isso $n$ vezes.
    \end{itemize}

\section*{Tabela comparativa}
    \begin{center}
        \begin{tabular}{|c|c|c|c|}
             \hline
             Tamanho do número & Tempo para gerar (LCG) & Tempo para gerar (PM)
             & Tempo para gerar (Xorshift) \\
             \hline
             40 bits   & 380.5ns  & 346ns    & 669.2ns \\
             56 bits   & 391.3ns  & 350.7ns  & 743.3ns \\
             80 bits   & 388.3ns  & 358.9ns  & 681.4ns \\
             128 bits  & 407.4ns  & 383.2ns  & 713.5ns \\
             168 bits  & 426ns    & 393.2ns  & 745.4ns \\
             224 bits  & 448.5ns  & 391.5ns  & 760.3ns \\
             256 bits  & 480.6ns  & 412.2ns  & 786.4ns \\
             512 bits  & 579.4ns  & 500.6ns  & 954.3ns \\
             1024 bits & 776.8ns  & 633.8ns  & 1227.8ns \\
             2048 bits & 1177.5ns & 1033.6ns & 1811.5ns \\
             4096 bits & 2002.5ns & 1974.3ns & 3060.5ns \\
             \hline
        \end{tabular}
    \end{center}
    A tabela acima mostra os resultados testados na geração de valores de cada
    um dos três algoritmos, com 10 milhões de \textit{rounds} cada um. Como é
    possível notar, conforme o número de bits cresce, mais tempo leva para
    gerar um número aleatório com aquela quantidade de bits e quem obteve
    melhor performance foi o algoritmo de Park-Miller.

\section*{Teste de aleatoriedade}
    \subsection*{Período}
    \begin{itemize}
        \item Os geradores de números pseudoaleatórios possuem um período, ou
            seja, ele terá uma sequência de números diferentes que foram
            gerados e, depois, volta a se repetir a sequência. O tamanho de tal
            sequência se chama período. Portanto, é necessário que o gerador
            contenha o maior período possível.
    
        \item No caso do LCG com $c \neq 0$, é possível conseguir um período de
            tamanho $m$, e isso acontece se e somente se for seguido o teorema
            de Hull-Dobell. Portanto, é necessário seguir três
            propriedades:\cite{Knuth1981}
        \begin{enumerate}
            \item $m$ e a constante $c$ são coprimos;
            \item $a-1$ é divisível por todos os fatores primos de $m$;
            \item $a-1$ é divisível por 4 se $m$ é divisível por 4.
        \end{enumerate}
        Os problemas que podem acontecer é, caso o valor do módulo seja do
            tamanho de uma palavra (32, 64, 128, etc.) é possível atingir
            grandes períodos, porém, apenas os bits mais significativos irão
            alcançar o período completo de $m$, enquanto os k bits menos
            significativos vão manter o período de $2^{k+1}$. Uma forma de
            resolver o problema anterior é usar um módulo que seja um valor
            primo diferente de 2, porém, está solução não irá produzir certos
            valores. A solução usada aqui é similar à implementada no
            \texttt{glibc}.
    
        \item Para o gerador de Parker-Miller com $m$ primo, o período pode ser
            até $m-1$ se o multiplicador $a$ é escolhido como um elemento
            primitivo dos inteiros módulo $m$. O estado inicial tem que ser
            entre 1 e $m-1$. A solução usada é similar à implementação usada
            pela função \texttt{minstd\_rand} do C++11.
    
        \item O Xorshift necessita escolher as constantes com mais cuidado a
            fim de atingir um melhor período. No caso de um estado de tamanho
            de 32 bits, o período pode ir a $2^{32}-1$ se houver deslocamento à
            esquerda por 13 bits, deslocamento à direita por 17 bits e
            deslocamento à esquerda por 5 bits, nesta ordem. O artigo
            original\cite{Marsaglia2003} define diversas constantes a utilizar
            e quando vale usá-las, neste caso, foi feito dois deslocamentos à
            esquerda por 24, um deslocamento à direita por 9 bits e um
            deslocamento à esquerda por 35 bits, nesta ordem. O módulo do
            resultado é encontrado para garantir que o número gerado tenha no
            máximo o tanto de bits que foi especificado.
    \end{itemize}
    
\section*{Comparações}
    O gerador de Park-Miller possui o funcionamento similar ao LCG, a diferença
    é que $c = 0$ no gerador de Park-Miller. Xorshift possui o funcionamento
    bem diferente de ambos, trabalhando, principalmente, com três operações: o
    ou-exclusivo e os deslocamentos de bits à esquerda e à direita. As
    constantes utilizadas no Xorshift e como os deslocamentos são feitos
    depende muito de quantos bits serão gerados, enquanto tanto o LCG quando o
    gerador Park-Miller definem o tanto de bits a serem gerados de acordo com o
    valor definido no módulo.

\bibliographystyle{acm}
\bibliography{02-prng}

\end{document}
